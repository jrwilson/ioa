\section{Related Work\label{related_work}}

Our development of ioa++ grew out of the challenges faced when trying to develop distributed systems with threads.
Lee elaborates on the problems faced when using threads for concurrency in~\cite{lee2006problem}.
These same concerns are relevant to trends towards increasingly concurrent programming to increase performance, given the leveling trend of clock frequencies and the advent of multi-core processors\cite{sutter2005software}.

The direct implementation of a formal model to simplify concurrent programming was inspired by the application of structured programming~\cite{dijkstra1968letters} to programming language design.
Structured programming simplifies development by allowing programmers to reason about their programs directly from the source code; regardless of whether of not the code it formally verified.
The I/O automata model of Lynch and Tuttle\cite{lynch1987hierarchical} was selected because it supports features essential to developing distributed systems~\cite{lynch1996distributed}.
%% Temporal logic proof techniques~\cite{manna1992temporal},~\cite{lamport1978time} also can be used to reason about I/O automata, towards developing \emph{correct} systems.

Incorporating a formal model of concurrency into a language to make concurrent programming easier is an established technique.
We consider three exemplars:  Communicating Sequential Processes, Actors, and threads.
The Communicating Sequential Processes (CSP) model~\cite{hoare1978communicating} is an extremely popular technique for modeling concurrent systems.
The fundamental concurrency control primitive in CSP is the \emph{rendezvous} which allows processes to synchronize and possibly exchange values.
CSP was influential in the design of the occam programming language~\cite{jones1987programming} and continues to influence programming language design, e.g., Go~\cite{go}.
An actor~\cite{agha1986actors} is a behavior that can perform a number of actions when receiving a message including (1) replacing itself with a new behavior, (2) sending messages to other actors, and (3) creating new actors.
The Actor model influenced the design of languages such as Erlang~\cite{armstrong1996concurrent} and Scala~\cite{odersky2004overview}.
The collective formalization of threads, e.g., semaphores~\cite{dijkstra1968cooperating}, monitors~\cite{hoare1974monitors}, all appear in popular libraries, e.g., POSIX threads~\cite{butenhof1997programming}, and languages, e.g., Java~\cite{christopher2000high}.

%% Frameworks for reusable asynchronous and concurrent modules
%%   Collective formalization of threads
%%     CORBA (asynchronous method invocation)/RMI/RPC (function call stack model, ~threads)
%%   CSP
%%   Actors

%% Comparing I/O automata to every model for asynchronous, concurrent, and distributed computing is beyond the scope of this paper, so we focus instead on exemplars.
%% The I/O automata model is based on state transitions which matches the imperative style that dominates modern concurrent programming.
%% This is similar to the motivation for UNITY~\cite{chandy1988parallel} and is in contrast to functional models such as the $\pi$-calculus~\cite{milner1992calculus} and actors~\cite{agha1986actors}.
%% The I/O automata model assumes independent state with explicit communication, like communicating sequential processes (CSP)~\cite{hoare1978communicating} but unlike threads~\cite{lee2006problem} or the UNITY model~\cite{chandy1988parallel}.

%% All models for asynchronous, concurrent, and distributed computing admit non-determinism.
%% Lee points out that the difficulties of thread-based programming come from a need to ``prune'' non-determinism when paired with shared state and adopts the view that software components should be deterministic with respect to concurrency save a few that introduce non-determinism in a controlled way~\cite{lee2006problem}.
%% Programs in UNITY are composed by concatenating program texts and resolving shared variables~\cite{chandy1988parallel}.
%% For this reason, they share many of the same concerns as threads under composition.
%% I/O automata can be viewed as taking an alternate approach by admitting non-determinism while prohibiting shared state.
%% In I/O automata, the sequential flow control seen with threads~\cite{lee2006problem} and CSP~\cite{hoare1978communicating} is replaced by the non-deterministic execution of conditional atomic actions which resembles execution in the UNITY model~\cite{chandy1988parallel}.
%% The ability to create new automata can be compared to actor creation in the actor model~\cite{agha1986actors}.
%% Actors communicate by name using a buffered mail system while communications in I/O automata are not buffered and anonymous.

% Other efforts to implement I/O automata have focused on simulation and verification~\cite{goldman1990distributed},~\cite{georgiou2009automated}.
% As observed in~\cite{georgiou2009automated}, a benefit of implementing distributed systems directly with I/O automata is the ability to reason about the behavior of a system or component directly from the source code using techniques from the I/O automata formalism.
% However, where~\cite{georgiou2009automated} introduces a new language and uses the Message Passing Interface (MPI)~\cite{gropp1999using} library for communication, ioa++ uses the existing C++ language and exposes native operating system services.

To our knowledge, the Spectrum Simulation System~\cite{goldman1990distributed} and the IOA language~\cite{garland2003ioa} are the only existing approaches that allow one to execute a system expressed as a collection of I/O automata.
Spectrum focuses on algorithm development through simulation.
IOA focuses on formal software development including both simulation and verification.

The ioa++ framework can be compared directly to the IOA language due to the availability of a compiler~\cite{tsai2002code},~\cite{tauber2004verifiable},~\cite{tauber2004compiling} that has been used to implement a number of protocols~\cite{georgiou2009automated}.
\cite{georgiou2009automated} lists six challenges when developing the compiler: ``Program structuring,'' ``IOA Programs and external services,'' ``Modeling procedure calls,'' ``Composing automata,'' ``Nondeterminism'', and ``Implementing datatypes.''
Programs written in IOA are compiled to Java and executed on hosts communicating using the MPI library.
IOA requires programs to presented in a ``node-channel'' form that allows them to be composed with built in ``mediators'' that implement external services such as MPI.
IOA lacks support for procedure calls which complicates modeling and interfacing with external services: especially when a procedure call blocks.
The IOA compiler composes automata statically using a ``composer'' resulting in a ``node automaton.''
The next action to be executed is determined by a schedule function that resolves the non-determinism in the schedule.

IOA and ioa++ represent two fundamentally different approaches to programming directly with I/O automata.
The development of a new language was necessary for IOA due to their focus on formal verification.
On the other hand ioa++ uses the existing C++ language and compilers.
IOA encapsulates external services using built in ``mediators'' while ioa++ exposes operating system services via file descriptors, relying on the programmer to ensure non-blocking operation.
IOA focuses on static composition to produce a single node program while ioa++ focuses on dynamic composition and allows concurrent execution.
Both IOA and ioa++ require the programmer to add code to schedule actions, which may be a common source of errors.
Scheduling in IOA is deterministic and the programmer provides the policy.
Scheduling in ioa++ is non-deterministic according to the global policy encoded by the controller.

%% CORBA thread model

%% %% Lee~\cite{lee2006problem} also identifies a number of strategies for coping with thread-based programming including libraries, patterns, programming languages, and coordination languages.
%% %% In~\cite{schmitd2000patterns}, Schmidt et al. describe patterns for developing asynchronous and concurrent objects and systems.
%% %% The Split-C~\cite{culler1993parallel} and Cilk~\cite{blumofe1995cilk} add features to the C programming language for multi-threaded computation.
%% %% The Erlang~\cite{armstrong1996concurrent} 

%% Ada

%% The 

%% Events

%% Lynch

%% Garland

%% Ken Goldman

%% proactor/reactor
%% Observer

