\documentclass[letterpaper, twocolumn]{article}

\usepackage{outline}

\begin{document}

\title{Threads Considered Harmful}
\author{Justin R. Wilson \and Christopher D. Gill}
\date{}

\maketitle

\begin{abstract}
Our ability to realize sophisticated distributed systems such as those in cyber-physical systems and pervasive computing is predicated on a solid model and implementation of asynchrony and concurrency.
Threads, the de facto standard in practice, are not a suitable foundation because 1) ad hoc communication among threads prohibits software reuse, 2) threads are difficult to treat formally and therefore \emph{very} difficult to treat informally, and 3) threads are ill-suited for the asynchronous semantics present in the systems we desire to build.
In this paper, we outline the semantics necessary for building sophisticated distributed systems and connect these to the I/O automata formal model.
We then describe an implementation of the I/O automata formal model and cover extensions to the model necessary for operating in a dynamic environment.
Finally, we conclude with an evaluation of our I/O automata framework.
\end{abstract}

\section{Introduction}

Advances in embedded systems and networking are paving the way for distributed systems of unprecedented sophistication.
Cyber-physical systems go beyond the traditional embedded system paradigm by explicitly modeling the dynamics of the physical system and communication network in the computational task.
The inclusion of wireless network interfaces in modern electronic devices is transforming our homes, offices, and hospitals into pervasive computing environments.
The interactions these systems have with the physical environment and its community of users require a degree of interoperability, configurability, adaptability, scalability, robustness, and security not found in existing systems.

One of the forces that prevents the development of any kind of sophisticated software system is accidental complexity~\cite{brooks_nsb}.
In the context of distributed systems, a major source of accidental complexity is a mismatch between the conceptual model of the system and the technology used to implement the system.
The mismatch between the conceptual model and implementation vehicle introduces defects due to translation errors, slows rapid prototyping, and retards iterative refinement.
While there is no ``silver bullet''~\cite{brooks_nsb}, reducing the accidental complexity derived from the mismatch between concept and implementation will increase productivity and make new levels of sophistication possible with existing levels of effort.

Threads are the de facto implementation vehicle for most of computing including distributed systems.
Threads dominate because modern processors, programming languages, and operating systems were all developed to support the thread model.
Furthermore, the widespread availability of thread libraries makes them a natural choice for system development.
While threads dominate in practice, using threads to develop systems has a number of significant drawbacks.

First, ad hoc communication among threads prohibits software reuse.
To illustrate this problem consider the challenge of developing an application with a graphical user interface that also uses a multi-threaded middleware library.
Graphical user interfaces often use a single-threaded event-based design, e.g., X, Swing.
The system integrator must write glue code that bridges the single-threaded event-based semantics to multi-threaded semantics.
Bridging between two thread models is achievable, however, the systems we desire to build will rely on many libraries each with their own thread model.
The importance of a unified approach to concurrency becomes apparent when one considers bridging between different threading models, e.g., event-based, reactor, proactor, active objects, and their attributes, e.g., priority, thread-creation, blocking vs. non-blocking I/O.

Second, threads are difficult to treat formally and therefore \emph{very} difficult to treat informally.
The correctness of programs is reasoned about either formally as part of formal software design and verification and/or informally during the debugging process.
Since most programs will never be formally verified, the correctness of most programs hinges solely on the developers ability to reason about the program.
Based on the historical example of structured programming~\cite{goto_considered_harmful}, there is a strong correlation between what is ``easy'' in the formal arena and what is ``easy'' in the informal arena, especially when the implementation mimics the formal model, e.g., structured programming languages.
It is well known that a formal treatment of threads is difficult due to the arbitrary interleaving of critical sections and ability to define critical sections of arbitrary scope~\cite{lee_threads}.
Thus, the correctness of most multi-threaded programs hinges on the developer's ability to explore arbitrary interleavings of critical sections, i.e., model checking, in their head.

Third, threads are ill-suited for the asynchronous semantics present in the systems we desire to build.
The distributed systems we desire to build are asynchronous at both the device and network levels.
Embedded devices must respond to asynchronous events in the environment.
The interrupt-driven TinyOS operating system for wireless sensor network motes is an exemplar in this area.
The fundamental communication primitive in real-world distributed systems is asynchronous message passing.
Realizing asynchrony in a threaded program requires the use of signals (interrupts) and/or (a)synchronous I/O multiplexing.
Signals are primitive, rigid, and difficult to use correctly.
I/O multiplexing, e.g., a select loop, is limited to operating system abstractions, e.g., file descriptors, and induces a reactive state machine structure that is difficult to understand and maintain.

Difficulties in developing asynchronous and concurrent software using threads has prompted a number of concurrent programming languages, e.g., Esterel, Erlang, Ptolemy, Autopipe, and language extensions, e.g., OpenMP, Cilk, Split-C.
Solutions that involve creating a new language or extending an existing language suffer from two main problems.
First, such solutions are often domain-specific and do not generalize to different applications, a necessity for heterogeneous distributed systems.
Furthermore, solutions not connected to a general-purpose formal model lack the important property that the complete system including the environment can be modeled as another component.
Second, such solutions often fail to become mainstream due to practical considerations.
For example, the combination of C and pthreads is much more popular than Erlang due to the popularity of C-like languages and C's connection with UNIX operating system.
Most language extensions are targeted at niche markets and therefore never standardized and implemented in mainstream compilers.

Thus, in order to build sophisticated distributed systems, we require a common semantics and model for building concurrent and asynchronous systems with straight-forward abstractions.
%% The model must be general to support the wide variety of devices and protocols necessary for cyber-physical and pervasive computing systems.
%% The model must facilitate the development of concurrent modules that can be reused.
%% The model must have a strong connection with a formal model to simplify the task of reasoning about individual components and complete systems.
%% The model must be implemented in such a way that it is easily approachable.
In section~\ref{system_model}, we expound upon the requirements of the model and connect the requirements with the I/O automata formal model.
Section~\ref{representation} describes how I/O automata are represented in our I/O automata framework.
In section~\ref{design}, we describe the design and implementation of our I/O automata framework.
Section~\ref{evaluation} provides an evaluation and discussion of our framework.
Section~\ref{related_work} contains an overview of related work and we offer our conclusions and thoughts for future work in section~\ref{conclusion}.

%% We provide a common semantics and model for building concurrent and asynchronous systems with straight-forward abstractions.

%%formal models, e.g., UNITY, I/O Automata,

%% Devices are asynchronous and concurrent.
%% Networks are asynchronous and concurrent.
%% Why not start with a model that is asynchronous and concurrent?

%% What inspires us:
%% \begin{itemize}
%%   \item CORBA - Remote Method Invocation (Remote Procedure Call with a ``this'' pointer).
%%     ``Let's make synchronous function calls over a network.''
%%     Assumes a thread of control.
%%     Asynchronous calls were added later (Did these result in obfuscation?)

%%   \item Patterns for asynchronous concurrency in a synchronous, thread-based world: Active objects, Reactor, Proactor, etc.
%%     Examples: X server, OS2 Presentation Manager, Java Swing

%%   \item AJAX - The modern Web is built on JavaScript + \emph{asynchronous} web page requests = AJAX.

%%   \item Michi Henning - ICE

%%   \item Steve Vinoski - Toolbox of programming languages

%% \end{itemize}

%% We provide a common semantics and model for building concurrent and asynchronous systems with straight-forward abstractions.

\section{System Model and Formalization\label{system_model}}

\begin{outline}
\item State
  \begin{outline}
  \item Local state only
  \item Shared state
    \begin{outline}
      \item Impossible in distributed systems
      \item Dangerous in local systems
    \end{outline}
  \end{outline}
\item Communication
  \begin{outline}
  \item Atomic asynchronous message passing
  \item Network sets size of atom (UDP)
  \item Can build reliable streams (TCP)
  \item Local equivalent is passing a value
  \item Model should lend itself to writing protocols
  \end{outline}
\item Asynchrony
  \begin{outline}
    \item Model must have natural support for asynchrony, i.e., event-based
    \item Leads to a more efficient implementation because changed state and enabled actions become obvious
  \end{outine}
\item Concurrency
  \begin{outline}
    \item Reason about systems using non-deterministic interleaving of atomic actions
    \item Model should admit implementations that execute concurrently
  \end{outline}
\item Dynamics
  \begin{outline}
    \item Configuration - Edges in graph of communicating components can change at run-time.
      \begin{outline}
      \item Already required in distributed settings
      \item Not addressed in formal models
      \end{outline}
    \item Extension - Nodes in graph of communicating components can change at run-time.
  \end{outline}
\end{outline}

I/O Automata
\begin{itemize}
  \item Compare with UNITY
  \item Compare with esterel
  \item Compare with pi calculus
  \item Compare with Ptolemy
\end{itemize}

\section{I/O Automata Representation and Programming Model\label{representation}}

Conclude with our recommendations for programming, highlight need for exploration.

\section{Design and Implementation\label{design}}

\begin{itemize}
  \item Problem
  \item Design forces
  \item Solution
  \item Consequences
\end{itemize}

\section{Evaluation and Discussion\label{evaluation}}

\begin{itemize}
\item Translation of automaton in \emph{Distributed Algorithms} to C++.
\item Simulate a protocol then replace with network components
\item Compare a protocol using: single-threaded event-based (select loop), multi-threaded, I/O automata
\item Show a buggy program and then apply an invariant to find the bug.
\end{itemize}

Bring up style mentioned in section \ref{representation}.

\section{Related Work\label{related_work}}

pthreads (?)
Edward Lee - The Problem with Threads
Herb Sutter - The Free Lunch is Over
Early work on concurrency

\section{Conclusion and Future Work\label{conclusion}}

\begin{itemize}
  \item We are going to use it to build the substrate.
  \item Speculate on moving down into operating system (device drivers would be easy, IPC including filesystem replaced by automata)
  \item Speculate on moving down into the hardware level (Local talent, Ivan Sutherland)
  \item We can take advantage of multi-core in a very straight-forward way
  \item New problems in scheduling (Pinning automata to processors to minimize actions that span two processors.  Maximum independent set.)
  \item We are non-blocking all the way.  Combine this with a deterministic implementation of the scheduler and model and there are serious opportunities for real-time.
\end{itemize}

\end{document}
