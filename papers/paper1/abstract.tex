\begin{abstract}
The ability to realize sophisticated distributed systems such as those found in cyber-physical, pervasive, or enterprise computing environments is predicated on a suitable model and implementation of asynchrony and concurrency.
The size and complexity of these systems implies that software re-use in their development, maintenance, and evolution is also essential.
Taken together, these requirements demand appropriate programming abstractions through which different forms of asynchrony and concurrency can be realized directly in reusable software.
% By ``different forms'' we mean single-threaded, multi-threaded, time triggered, event triggered, etc.

To that end, this paper presents the \emph{ioa++ framework} for developing distributed systems, which is the primary contribution of this research.
We develop a component model for asynchronous and concurrent systems based on I/O automata and dynamic composition.
We describe the design and implementation of the component model and show how ioa++ can be used to develop protocols and other forms of distributed software.

Our experiences demonstrate that ioa++ can help simplify developing distributed systems, by supporting reasoning about asynchronous and concurrent programs directly from the source code using the I/O automata formal model.
Common concurrency and interaction semantics allow simple modules to be aggregated into complex systems without building adapters that convert from one concurrency model to another.
Whereas reasoning about the correctness of modules in a threaded program requires considering interactions among the various threads, reasoning about the correctness of modules in ioa++ is limited to the modules themselves and a well-defined set of composition rules.

We evaluate concurrency in ioa++ through a series of micro-benchmarks which confirm that concurrent execution is limited only by the interactions of the automata that comprise the system and the overhead of synchronization.
\end{abstract}
