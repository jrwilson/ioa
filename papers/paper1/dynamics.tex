\section{Dynamics\label{dynamics}}

%% All components have outputs for creating, binding, unbinding, and destroying and inputs for receiving results that are permanently bound to ``the system.''
%% System calls follow a request-reply system where a component sends a request for a create, bind, unbind, or destroy, the system receives the request, the system processes the request, the system sends a reply, and the component receives the reply.

%Events can be bound and unbound at run-time.
%% The I/O automata model is static, i.e., the set of automata and their interactions are fixed.
% Components can be created and destroyed at run-time.

However, the I/O automata model assumes a static configuration while the proposed system requires the ability to deal with dynamic configurations, i.e., creating, binding, unbinding, and destroying.

\subsection{I/O Automata and Dynamic Configuration}

A standard technique for modeling dynamic modules in I/O automata is to assume they exist but are inactive.
Dynamic automata can then be activated and deactivated using inputs.
We will use a similar technique to model dynamic configurations.

We introduce a \emph{system automaton} that records the configuration of all automata in the system.
We differentiate between automaton types and automaton instances.
Every automaton instances has a unique \emph{automaton identifier (aid)}.
The state of the system automaton consists of the set of aids $A$ indicating what automata exist and a set of binding records $B$ indicating the relationship between output and input actions.
A binding record $(out, output, in, input)$ is tuple consisting of the aid of the output automaton $out$, the name of the output action $output$, the aid of the input automaton $in$, and the name of the input action $input$.

The configuration contained in set of binding records must be a valid static configuration to ensure that the behavior of the complete system remains analyzable under the I/O automata model.
First, the output automaton and input automaton must exist $\forall (out, output, in, input) \in B: out \in A \land in \in A$.
% TODO:  Binding must not exist.
Second, an input can only be bound to one output.
Thus, a given $in$ and $input$ combination can appear at most once in $B$.
Third, an output cannot be bound to an input in the same automaton.
Thus, $out \neq in$ for every $(out, output, in, input)$ in $B$.
Fourth, an output cannot be bound to two inputs in the same automaton.
Thus, $in_1 \neq in_2$ for all $(out, output, in_1, input_1), (out, output, in_2, input_2)$ in $B$.

The system automaton contains input actions for creating an automaton, binding an output to an input, unbinding an output and an input, and destroying an automaton and output actions that indicate the result of a create, bind, unbind, or destroy.
All automata are composed with the system automaton and contain outputs for creating, binding, unbinding, and destroying and inputs for receiving results.
Systems calls (create, bind, unbind, destroy) resemble a request-response protocol where an automaton sends a request, the system automaton receives the request, the system processes the request, the system automaton then sends the result, and the automaton receives the result.
This has the important consequence that \emph{configuration is asynchronous}.
Destroying an automaton has the effect of removing all of its bindings.
An automaton that destroys itself can therefore cause the system to deliver unbound and destroyed events without an unbind or destroy.

Using the system automaton as a bookkeeping mechanism, we can relate a statically composed system to a dynamically configured one.
The key idea is to show that actions belong to automaton $a$ are only executed when $a \in A$ and that an output action is only executed when $B$ contains all of bindings associated with the static configuration.


\subsection{Extending I/O Automata for Reflection}

Automata are endowed with additional inputs driven by system outputs indicating when an action has been bound and unbound.
Automata might use the bound input to enable processing, i.e., someone is listening so I'll start producing data.
The unbound input is useful for cleaning data structures and inhibiting further processing, i.e., no one is listening so I'll stop producing data.
The bound and unbound inputs are useful but they are also asynchronous, thus, an automaton cannot trust the status reported by the last bound or unbound for use in, for example, a precondition.

The asynchronous nature of bound and unbound requires a primitive that can examine the contents of $B$ instantaneously.
Thus, we make one exception to the rule of local state that says automata can examine the contents of $B$ to determine their current configuration.
This is permissible so long as concurrent implementations provide multiple-reader/single-writer semantics to $B$.

\subsection{Equivalence}

Extending the I/O automata model for dynamics and reflection is useless unless we can relate the extensions back to the original model or establish equivalence.
To do this, we will take advantage of the fair traces (?) property of I/O automata~\cite{distributed_algorithms}.
%% We accept an execution trace in the dynamic setting if it is a fair trace of the static model.

%% Modeling a system with I/O automata implies a fixed set of automata given by $A_0$ and set of binding records $B_0$.

Start with a trace of dynamic $T_d$.
Remove system actions from $T_d$ so $T_d'$.
We accept $T_d'$ if it is a fair trace of the static model.

Let $A_0$ be the set of automata and $B_0$ be the set of binding records in the static model.
If you want to prove correspondence, you must prove that eventually $A = A_0$ and $B = B_0$.
We also need to show that no output is executed before it is fully bound.

%% supports the actions of creating an automaton, binding an output to an input, unbinding an output and an input, and destroying an automaton.
%% Upon creation, All automata are bound
%% The four operations necessary for dynamic configuration are \emph{create}, \emph{bind}, \emph{unbind}, and \emph{destroy}.

%% \begin{outline}
%% \item State
%%   \begin{outline}
%%   \item There is no shared state in a 
%%   \item Local state only
%%   \item Shared state
%%     \begin{outline}
%%       \item Impossible in distributed systems
%%       \item Dangerous in local systems
%%     \end{outline}
%%   \end{outline}
%% \item Communication
%%   \begin{outline}
%%   \item Atomic asynchronous message passing
%%   \item Network sets size of atom (UDP)
%%   \item Can build reliable streams (TCP)
%%   \item Local equivalent is passing a value
%%   \item Model should lend itself to writing protocols
%%   \end{outline}
%% \item Asynchrony
%%   \begin{outline}
%%     \item Model must have natural support for asynchrony, i.e., event-based
%%     \item Leads to a more efficient implementation because changed state and enabled actions become obvious
%%   \end{outline}
%% \item Concurrency
%%   \begin{outline}
%%     \item Reason about systems using non-deterministic interleaving of atomic actions
%%     \item Model should admit implementations that execute concurrently
%%   \end{outline}
%% \item Dynamics
%%   \begin{outline}
%%     \item Configuration - Edges in graph of communicating components can change at run-time.
%%       \begin{outline}
%%       \item Already required in distributed settings
%%       \item Not addressed in formal models
%%       \end{outline}
%%     \item Extension - Nodes in graph of communicating components can change at run-time.
%%   \end{outline}
%%   \item Reflection
%% \end{outline}

%% I/O Automata
%% \begin{itemize}
%%   \item Compare with UNITY
%%   \item Compare with esterel
%%   \item Compare with pi calculus
%%   \item Compare with Ptolemy
%% \end{itemize}
