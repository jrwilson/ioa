\section{Evaluation and Discussion\label{evaluation}}

The goal of this section is to provide a preliminary exploration of concurrent execution for I/O automata and establish a baseline for system overhead.

Time spent executing a system of automata falls into four categories.
\emph{Ioa++} time indicates time spent executing framework code.
\emph{Thread} time indicates time spent executing (blocking) calls to the pthreads library.
\emph{Scheduling} time indicates time spent for calls into the framework to schedule an action.
\emph{User} time indicates the time spent executing automaton code.
The \emph{total} time is the sum of the four categories.

\paragraph{Experimental design.}
An automaton $R$ contains an input, output, and internal action.
The output and internal action execute a routine whose complexity is proportional to the parameter $N$.
The automaton schedules the internal action with probability $\rho$ and schedules the output action otherwise.
The automaton executes a total of the 1,000 local actions.
The system to be executed consists of two $R$ automata composed so the output action of one is composed with the input action of another.

The number of threads, the parameter $N$, and the parameter $\rho$ constitute the configuration parameters for the experiment.
The internal actions are independent and can be executed concurrently by independent threads.
Increasing $N$ increases user time.
When $\rho = 0$, the internal actions are never scheduled and the output actions are always scheduled.
Consequently, no actions are executed concurrently.
Conversely, when $\rho = 1$, the internal actions are always scheduled and every action can be executed concurrently.

\paragraph{Results.}
We executed the described system for $N = 10^{\{1, 2, 3, 4, 5, 6\}}$ and $\rho$ from $0.0$ to $1.0$ in $0.1$ increments.
The results are shown in Figures~\ref{plot1}, \ref{plot10}, \ref{plot100}, \ref{plot1000}, \ref{plot10000}, \ref{plot100000}, and \ref{plot1000000}.


\begin{figure}
\center
\includegraphics[width=\textwidth]{plot1}
\label{plot1}
\end{figure}

\begin{figure}
\center
\includegraphics[width=\textwidth]{plot10}
\label{plot10}
\end{figure}

\begin{figure}
\center
\includegraphics[width=\textwidth]{plot100}
\label{plot100}
\end{figure}

\begin{figure}
\center
\includegraphics[width=\textwidth]{plot1000}
\label{plot1000}
\end{figure}

\begin{figure}
\center
\includegraphics[width=\textwidth]{plot10000}
\label{plot10000}
\end{figure}

\begin{figure}
\center
\includegraphics[width=\textwidth]{plot100000}
\label{plot100000}
\end{figure}

\begin{figure}
\center
\includegraphics[width=\textwidth]{plot1000000}
\label{plot1000000}
\end{figure}

0.000357291 0.000767172
0.000359442 0.000769291
