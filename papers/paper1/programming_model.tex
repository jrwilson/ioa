\section{I/O Automata Representation and Programming Model\label{representation}}

Conclude with our recommendations for programming, highlight need for exploration.

%% \begin{outline}
%% \item State
%%   \begin{outline}
%%   \item There is no shared state in a 
%%   \item Local state only
%%   \item Shared state
%%     \begin{outline}
%%       \item Impossible in distributed systems
%%       \item Dangerous in local systems
%%     \end{outline}
%%   \end{outline}
%% \item Communication
%%   \begin{outline}
%%   \item Atomic asynchronous message passing
%%   \item Network sets size of atom (UDP)
%%   \item Can build reliable streams (TCP)
%%   \item Local equivalent is passing a value
%%   \item Model should lend itself to writing protocols
%%   \end{outline}
%% \item Asynchrony
%%   \begin{outline}
%%     \item Model must have natural support for asynchrony, i.e., event-based
%%     \item Leads to a more efficient implementation because changed state and enabled actions become obvious
%%   \end{outline}
%% \item Concurrency
%%   \begin{outline}
%%     \item Reason about systems using non-deterministic interleaving of atomic actions
%%     \item Model should admit implementations that execute concurrently
%%   \end{outline}
%% \item Dynamics
%%   \begin{outline}
%%     \item Configuration - Edges in graph of communicating components can change at run-time.
%%       \begin{outline}
%%       \item Already required in distributed settings
%%       \item Not addressed in formal models
%%       \end{outline}
%%     \item Extension - Nodes in graph of communicating components can change at run-time.
%%   \end{outline}
%%   \item Reflection
%% \end{outline}

%% I/O Automata
%% \begin{itemize}
%%   \item Compare with UNITY
%%   \item Compare with esterel
%%   \item Compare with pi calculus
%%   \item Compare with Ptolemy
%% \end{itemize}
