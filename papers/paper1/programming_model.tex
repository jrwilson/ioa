\section{I/O Automata Programming Model\label{programming_model}}

In this section, we show how I/O automata can be used to construct programs using ioa++.
Section~\ref{representation} describes how I/O automata are represented in the C++ language.
We then discuss the mechanisms available for dynamic composition in Section~\ref{dynamic_composition}.
Section~\ref{constellations} concludes with techniques for managing dynamic constellations of automata.

\subsection{I/O Automata Representation\label{representation}}

All automata inherit from the common base class ioa::automaton which implements the system action interface.
\ifjournal
The following code declares the asynch\_lcr\_automaton:
\begin{lstlisting}
template <typename UID>
class asynch_lcr_automaton :
  public ioa::automaton
{
  ...
};
\end{lstlisting}
The asynch\_lcr\_automaton is declared as a C++ template where the UID template parameter must model the \emph{unique identifier} concept.
The metaprogramming facilities of the C++ language aid in the development of general purpose reusable components.
\fi

\paragraph*{Declaring and initializing state variables}
The state variables of an automaton are expressed as member variables of its corresponding class or class template.
\ifjournal
The four state variables of the asynch\_lcr\_automaton include the UID of the automaton, a queue of outgoing messages, and flags for reporting a change in status.
\begin{lstlisting}
private:
  const UID m_u;
  std::queue<UID> m_send;
  bool m_report;
  bool m_leader;
\end{lstlisting}
\fi
Member variables are declared private, which is in keeping with the notion that the state of each automaton is independent, and are initialized using a constructor.
The constructor can also bootstrap the scheduler using the ioa::schedule function discussed below.
\ifjournal
For the asynch\_lcr\_automaton, the flags are both set initially false and the UID is initialized and added to the send queue.
\begin{lstlisting}
public:
  asynch_lcr_automaton (const UID& u) :
    m_u (u),
    m_report (false),
    m_leader (false)
  {
    m_send.push (m_u);
    schedule ();
  }
\end{lstlisting}
The constructor also bootstraps the scheduler by scheduling all enabled local actions using the ioa::schedule function discussed below.
\fi

\paragraph*{Declaring/defining actions}
Actions are member variables that model the \emph{action} concept.
Actions contain methods for preconditions, effects, and scheduling along with \emph{traits} that indicate the action's type (input, output, internal), value status, value type, parameter status, and parameter type.
These traits are used to guide the execution of the methods by the model and to supply values and parameters as necessary.
The traits are also used to check that input actions are bound to output actions with the same value status and type.
Bindings are checked for type compatibility at compile time using the generic programming facilities of C++.

A number of wrappers and macros are provided by ioa++ for declaring actions.
%%To facilitate declaring actions, a number of wrappers and macros corresponding to the action types listed in section~\ref{practical} are available.
Local actions are declared using three functions corresponding respectively to precondition, effect, and scheduling.
%%The macros assume the functions are named action\_name\_precondition, action\_name\_effect, and action\_name\_schedule.
\ifjournal
For example, the send action of the asynch\_lcr\_automaton is declared:
\begin{lstlisting}
private:
  bool send_precondition () const { ... }
  UID send_effect () { ... }
  void send_schedule () const { ... }
public:
  V_UP_OUTPUT (asynch_lcr_automaton, send, UID);
\end{lstlisting}
\fi
Input actions are declared similarly except for the absence of a precondition.
The precondition and scheduling functions are not allowed to change the state of the automaton and are consequently declared const.
The macros for actions that are subject to binding by other automata are declared public.

\paragraph*{Scheduling}
Automata submit actions to the scheduler using the ioa::schedule function.
The ioa::schedule function can be called at any time.
However, automata are generally easier to read if scheduling calls are relegated to the scheduling functions.
A common pattern is to schedule all local actions using one member function where each action is guarded by its precondition as described in section~\ref{scheduling}.
Another common pattern is to have all scheduling functions dispatch to a single scheduling function to avoid replicating scheduling logic.
\ifjournal
For the asynch\_lcr\_automaton, the scheduling function is:
\begin{lstlisting}
  void schedule () const {
    if (send_precondition ()) {
      ioa::schedule (&asynch_lcr_automaton::send);
    }
    if (leader_precondition ()) {
      ioa::schedule (&asynch_lcr_automaton::leader);
    }
  }
\end{lstlisting}
The various scheduling functions in the asynch\_lcr\_automaton and the constructor all dispatch to this common member function.
\fi

\subsection{Dynamics\label{dynamic_composition}}

\paragraph*{Dynamic composition}
%% The asynch\_lcr\_automaton is a reusable component designed to be composed in the context of a larger system.
Dynamic composition is accomplished by creating child automata and bindings via ioa::automaton\_manager and ioa::binding\_manager objects.
\ifjournal
The tcp\_lcr\_automaton discussed in section~\ref{case_study} creates a asynch\_lcr\_automaton and binds to it with the following code:
\begin{lstlisting}
ioa::automaton_manager<asynch_lcr_automaton<uuid> >*
lcr = ioa::make_automaton_manager (this,
  ioa::make_generator<asynch_lcr_automaton<uuid> > (
    m_u));
ioa::make_binding_manager (this,
  lcr, &asynch_lcr_automaton<uuid>::send,
  &m_self, &tcp_lcr_automaton::send_lcr);
ioa::make_binding_manager (this,
  &m_self, &tcp_lcr_automaton::receive_lcr,
  lcr, &asynch_lcr_automaton<uuid>::receive);
ioa::make_binding_manager (this,
  lcr, &asynch_lcr_automaton<uuid>::leader,
  &m_self, &tcp_lcr_automaton::leader_lcr);
ioa::make_binding_manager (this,
  &m_self, &tcp_lcr_automaton::init_lcr,
  lcr, &asynch_lcr_automaton<uuid>::init);
\end{lstlisting}
\fi
%%Child automata are managed by ioa::automaton\_manager objects which can be created via the ioa::make\_automaton\_manager function.
An automaton manager object requires an ioa::automaton object for its system action implementation and a factory for allocating child automata.
%%Bindings are managed by ioa::binding\_manager objects which can be created via the ioa::make\_binding\_manager function.
A binding manager object requires an ioa::automaton object for its system action implementation, a manager object for the output automaton, a reference to the output action, a manger object for the input automaton, a reference to the input action, and parameters for the actions if required.
The ioa::automaton\_manager class and ioa::binding\_manager class contain methods for destroying and unbinding respectively.
\ifjournal
The m\_self object of the example is an ioa::handle\_manager which can be used to bind to automata that already exist.
From the example, the tcp\_lcr\_automaton binds the asynch\_lcr\_automaton to itself.
The wary reader might notice that the ioa::automaton\_manager and ioa::binding\_managers in the example are dynamically allocated but never recorded and thus never freed.
Both classes are managed by the ioa::automaton and automatically deallocated when the automaton or binding they manage is destroyed or unbound respectively.
\fi

\paragraph*{Binding predicates}
Recall that automata require the ability to inspect the current set of bindings to adhere to the I/O automata model.
The binding predicate supplied by ioa++ is the ioa::binding\_count function which returns the number of other actions bound to the specified action.
The value returned by ioa::binding\_count is zero for internal actions, zero or one for input actions, and non-negative for output actions.
Most often, ioa::binding\_count is used in the precondition of outputs to ensure that the value generated by the output will be received if produced.
\ifjournal
This type of usage can be seen in the send action of the asynch\_lcr\_automaton:
\begin{lstlisting}
bool send_precondition () const {
  return !m_send.empty () &&
    ioa::binding_count (&asynch_lcr_automaton::send)
      != 0;
}
\end{lstlisting}
\fi

\subsection{Managing Dynamic Constellations\label{constellations}}

In this section, we introduce techniques and principles for managing dynamic constellations of automata and bindings that go beyond the ``nuts and bolts'' presented in the previous section.
This list is not exhaustive but rather serves as a distillation of our experiences so far working with I/O automata in ioa++.

\paragraph*{System action events}
Determining the outcome of a system action event (e.g., that a child automaton was created) is necessary when there are constraints on the order in which system actions must occur.
For example, the Binding Manager mentioned in section~\ref{system_action_section} waits until the automata that supply the output action and input action are created, before attempting to bind.
Similarly, an automaton might compose with an automaton that it does not manage and therefore may expect the automaton to exist before attempting to bind.
To enable such behavior, user automata can observe ioa::automaton\_manager objects and ioa::binding\_manager objects to receive events generated upon the completion of a system action.

\paragraph*{Error reporting}
The event semantics of I/O automata provide a natural mechanism for reporting errors.
An automaton that encounters an error can use an output action to signal the error.
The automaton can continue processing if the error is transient or stop if the error is permanent.
In either case, the error is propagated to the automaton that is using the automaton that encountered an error.
Using this technique, errors can be propagated up the hierarchy and resolved at an appropriate level.
The two techniques for handling an error are either to reset or to replace the portion of the constellation that encountered the error.

\paragraph*{Handling errors by resetting}
An automaton that can be reset has one or more output actions for reporting errors and one or more input actions for resetting.
Automaton designers should strive to write automata that are capable of being reset as this is the simplest error handling strategy.

\paragraph*{Handling errors by replacement}
The non-deterministic and asynchronous nature of system actions complicates replacing an automaton.
To replace a child automaton, the parent automaton must destroy the child (thus dissolving all bindings with the child), create a new child, and bind to the new child.
Recall that system actions are asynchronous and can be executed in any order.
Thus, the user might might request the actions in one order, e.g., destroy, create, bind, but the system may execute them in a different order, e.g., create, bind, destroy.
If the sequence executed by the system automaton is (create, bind, destroy) then the bind will fail because the user automaton's actions are still bound to the old child.
Note that executing the bind action before the create will also fail but this case is prevented by the ioa::binding\_manager class.
Thus, a user automaton must receive notification that the first child has been destroyed before (creating and) binding to the second child.
A general solution is to observe the automaton and wait for it to be destroyed.

If all of the bindings of a child automaton are with its parent, then the child can be replaced using parameters.
Instead of binding the new child to the exact same actions of the parent that are being used by the old child, which is the root problem in the preceding scenario, the new child is bound to actions parameterized by the new child.
The parameters serve to differentiate the actions and obviate the need to wait for the old child to be destroyed.
To use this technique, the parent remembers the parameter associated with the current child and becomes unresponsive to actions not bearing that parameter.
